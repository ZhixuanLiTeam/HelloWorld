\documentclass[10pt]{article}

\usepackage{amsmath}
\usepackage{pstricks,pst-eps}
\usepackage{mathrsfs}
\usepackage{amssymb}



\pagestyle{empty}
\begin{document}
\begin{TeXtoEPS}

  \begin{pspicture}(-0.5,-0.8)(12.5,4)
  %原三角剖的
  \psdots*(0,1)(1,0)(2,1.7)(3,0)(3.5,1)
  \psline[linewidth=.5pt](0,1)(1,0)
  \psline[linewidth=.5pt](0,1)(2,1.7)
  \psline[linewidth=.5pt](1,0)(2,1.7)
  \psline[linewidth=.5pt](1,0)(3,0)
  \psline[linewidth=.5pt](2,1.7)(3,0)
  \psline[linewidth=.5pt](2,1.7)(3.5,1)
  \psline[linewidth=.5pt](3,0)(3.5,1)
  %增加的点
  \psdots[dotstyle=square*,dotsize=0.15](1,1.35)(1.5,0.81)(2.5,0.81)(2,0)
  %增加的线
  \psline[linewidth=.5pt](1,1.35)(1.5,0.81)
  \psline[linewidth=.5pt](0,1)(1.5,0.81)
  \psline[linewidth=.5pt](1.5,0.81)(2.5,0.81)
  \psline[linewidth=.5pt](1.5,0.81)(2,0)
  \psline[linewidth=.5pt](2,0)(2.5,0.81)
  \psline[linewidth=.5pt](2.5,0.81)(3.5,1)

  
  
   % the arrows for the flow map
  \psline[arrowsize=0.2]{->}(4,0.5)(6.4,0.5)
  \psline[arrowsize=0.2]{->}(4,1.6)(6.7,2)
  \uput[u](5.1,0.5){$\varphi_{t_n}^{+k}$}
  \uput[d](5.2,2.6){$\varphi_{t_n}^{+k}$}
  

  
  
  
  %原三角剖分拉长以后
  \psdots(7,1)(8.3,0)(10,3.4)(12,0)(11.8,2)
%  \psline[linewidth=.5pt](7,1)(8.3,0)
%  \psline[linewidth=.5pt](8.3,0)(10,3.4)
%  \psline[linewidth=.5pt](7,1)(10,3.4)
%  \psline[linewidth=.5pt](12,0)(10,3.4)
%  \psline[linewidth=.5pt](8.3,0)(12,0)
%  \psline[linewidth=.5pt](10,3.4)(11.8,2)
%  \psline[linewidth=.5pt](12,0)(11.8,2)
  %增加的点
  \psdots[dotstyle=square*,dotsize=0.15](8.3,2.5)(9.2,1.5)(10,-0.3)(10.7,1.4)
  \psline[linewidth=.5pt](7,1)(8.3,2.5)
  \psline[linewidth=.5pt](7,1)(8.3,0)
  \psline[linewidth=.5pt](7,1)(9.2,1.5)
  \psline[linewidth=.5pt](9.2,1.5)(8.3,0)
  \psline[linewidth=.5pt](8.3,2.5)(9.2,1.5)
  \psline[linewidth=.5pt](8.3,0)(10,-0.3)
  \psline[linewidth=.5pt](10,-0.3)(9.2,1.5)
  \psline[linewidth=.5pt](10,-0.3)(12,0)
  \psline[linewidth=.5pt](10,-0.3)(10.7,1.4)
  \psline[linewidth=.5pt](10.7,1.4)(12,0)
  \psline[linewidth=.5pt](12,0)(11.8,2)
  \psline[linewidth=.5pt](11.8,2)(10.7,1.4)
  \psline[linewidth=.5pt](8.3,2.5)(10,3.4)
  \psline[linewidth=.5pt](9.2,1.5)(10,3.4)
  \psline[linewidth=.5pt](9.2,1.5)(10.7,1.4)
  \psline[linewidth=.5pt](10,3.4)(11.8,2)
  \psline[linewidth=.5pt](10,3.4)(10.7,1.4)
  
  %特殊点标记
  \uput[dr](-0.5,1.0){$\overleftarrow{p_0}$}
  \uput[dr](0.8,0){$\overleftarrow{p_1}$}
  \uput[dr](1.5,2.3){$\overleftarrow{p_2}$}
  \uput[dr](3.0,0){$\overleftarrow{p_3}$}
  \uput[dr](3.5,1.2){$\overleftarrow{p_4}$}
  
  \uput[dr](6.7,1.0){$p_0$}
  \uput[dr](8.1,0){$p_1$}
  \uput[dr](9.8,4.0){$p_2$}
  \uput[dr](12,0){$p_3$}
  \uput[dr](12,2.2){$p_4$}

  \uput[u](1,1.35){$\overleftarrow{q_1}$}
  \uput[d](1.5,0.81){$\overleftarrow{q_2}$}
  \uput[d](2.5,0.81){$\overleftarrow{q_3}$}
  \uput[d](2,0){$\overleftarrow{q_4}$}
  
  \uput[u](8.3,2.5){$q_1$}
  \uput[d](9.2,1.4){$q_2$}
  \uput[d](10.8,1.3){$q_3$}
  \uput[d](10,-0.3){$q_4$}
  \end{pspicture}

\end{TeXtoEPS}
\end{document}
