\documentclass[10pt]{article}

\usepackage{amsmath}
\usepackage{pstricks,pst-eps}
\usepackage{mathrsfs}
\usepackage{amssymb}



\pagestyle{empty}
\begin{document}
\begin{TeXtoEPS}

  \begin{pspicture}(-0.2,-0.1)(9.5,2.3)
  %原三角剖的
  \psdots*(0,0)(1,0)(2,0)(0.2,1)(1.8,1)(1,1.2)(0.5,2)(1.5,1.8)
  \psline[linewidth=.5pt](0,0)(1,0)
  \psline[linewidth=.5pt](1,0)(2,0)
  \psline[linewidth=.5pt](2,0)(1.8,1)
  \psline[linewidth=.5pt](1.8,1)(1.5,1.8)
  \psline[linewidth=.5pt](1.5,1.8)(0.5,2)
  \psline[linewidth=.5pt](0.5,2)(0.2,1)
  \psline[linewidth=.5pt](1,1.2)(0.2,1)
  \psline[linewidth=.5pt](0.2,1)(1.8,1)
  \psline[linewidth=.5pt](1.8,1)(1,1.2)
  \psline[linewidth=.5pt](0.2,1)(1,0)
  \psline[linewidth=.5pt](1,0)(1.8,1)
  \psline[linewidth=.5pt](0.5,2)(1,1.2)
  \psline[linewidth=.5pt](1,1.2)(1.5,1.8)
  \psline[linewidth=.5pt](0.2,1)(0,0) 
  %增加的点
  %\psdots[dotstyle=square*,dotsize=0.15](1,1.35)(1.5,0.81)(2.5,0.81)(2,0)

  
  
   % the arrows for the flow map
  \psline[arrowsize=0.2]{->}(3,1)(6,1)
  \uput[u](4.5,1){$\chi_{n+1}$}
 
  

  \psdots*(7,0)(8,0)(9,0)(7.2,1)(8.8,1)(8,1.2)(7.5,2)(8.5,1.8)
  \psline[linewidth=.5pt](7,0)(8,0)
  \psline[linewidth=.5pt](8,0)(9,0)
  \psline[linewidth=.5pt](9,0)(8.8,1)
  \psline[linewidth=.5pt](8.8,1)(8.5,1.8)
  \psline[linewidth=.5pt](8.5,1.8)(7.5,2)
  \psline[linewidth=.5pt](7.5,2)(7.2,1)
  \psline[linewidth=.5pt](8,1.2)(7.2,1)
  \psline[linewidth=.5pt](8,1.2)(8,0)
  \psline[linewidth=.5pt](8.8,1)(8,1.2)
  \psline[linewidth=.5pt](7.2,1)(8,0)
  \psline[linewidth=.5pt](8,0)(8.8,1)
  \psline[linewidth=.5pt](7.5,2)(8,1.2)
  \psline[linewidth=.5pt](8,1.2)(8.5,1.8)
  \psline[linewidth=.5pt](7.2,1)(7,0) 
       
  \uput[l](0.2,1){$p_1$}
  \uput[r](1.8,1){$p_2$}
  \uput[u](1,1.2){$p_3$}
  \uput[d](1,0){$p_4$}
  \uput[u](8,1.2){$p_3$}
  \uput[l](7.2,1){$p_1$}
  \uput[r](8.8,1){$p_2$}
  \uput[d](8,0){$p_4$}
  \end{pspicture}

\end{TeXtoEPS}
\end{document}
