\documentclass[10pt]{article}

\usepackage{amsmath}
\usepackage{pstricks,pst-eps}
\usepackage{mathrsfs}
\usepackage{amssymb}



\pagestyle{empty}
\begin{document}
\begin{TeXtoEPS}

  \begin{pspicture}(-0.2,-0.1)(9,2.3)
  %原三角剖的
  \psdots*(0,0.5)(1,0)(0,1.5)(2,1.6)(1,2.2)(0.8,1)(1,1)(2,0.5)
  \psline[linewidth=.5pt](0,0.5)(1,0)
  \psline[linewidth=.5pt](1,0)(2,0.5)
  \psline[linewidth=.5pt](2,0.5)(2,1.6)
  \psline[linewidth=.5pt](2,1.6)(1,2.2)
  \psline[linewidth=.5pt](1,2.2)(0,1.5)
  \psline[linewidth=.5pt](0,1.5)(0,0.5)
  \psline[linewidth=.5pt](0.8,1)(1,1)
  \psline[linewidth=.5pt](0.8,1)(0,0.5)
  \psline[linewidth=.5pt](0.8,1)(1,0)
  \psline[linewidth=.5pt](0.8,1)(0,1.5)
  \psline[linewidth=.5pt](1,1)(1,0)
  \psline[linewidth=.5pt](1,1)(2,0.5)
  \psline[linewidth=.5pt](1,1)(2,1.6)
  \psline[linewidth=.5pt](0.8,1)(0,1.5)
  \psline[linewidth=.5pt](0.8,1)(1,2.2)
  \psline[linewidth=.5pt](1,1)(2,1.6)
  \psline[linewidth=.5pt](1,1)(1,2.2)
  %增加的点
  %\psdots[dotstyle=square*,dotsize=0.15](1,1.35)(1.5,0.81)(2.5,0.81)(2,0)

  
  
   % the arrows for the flow map
  \psline[arrowsize=0.2]{->}(3,1)(6,1)
  \uput[u](4.5,1){$\chi_{n+1}$}
 
  

  \psdots*(7,0.5)(8,0)(7,1.5)(9,1.6)(8,2.2)(7.9,1)(9,0.5)
  \psline[linewidth=.5pt](7,0.5)(8,0)
  \psline[linewidth=.5pt](8,0)(9,0.5)
  \psline[linewidth=.5pt](9,0.5)(9,1.6)
  \psline[linewidth=.5pt](9,1.6)(8,2.2)
  \psline[linewidth=.5pt](8,2.2)(7,1.5)
  \psline[linewidth=.5pt](7,1.5)(7,0.5)  
  \psline[linewidth=.5pt](7,0.5)(7.9,1)
  \psline[linewidth=.5pt](8,0)(7.9,1)
  \psline[linewidth=.5pt](9,0.5)(7.9,1)
  \psline[linewidth=.5pt](9,1.6)(7.9,1)
  \psline[linewidth=.5pt](8,2.2)(7.9,1)
  \psline[linewidth=.5pt](7,1.5)(7.9,1)  
       
  \uput[l](0.8,1){$p_1$}
  \uput[r](1,1){$p_2$}
  \uput[d](1,0){$p_3$}
  \uput[dr](7.9,1){$q$}

  \end{pspicture}

\end{TeXtoEPS}
\end{document}
