
\chapter{结论}
\section{方法总结和创新点}
多相流研究的迅速发展,对界面追踪算法的精度和效率提出了更高的要求.
MARS方法用几何与拓扑的工具手段来处理几何拓扑问题,
最大化地保留了流体的拓扑信息,不仅在计算效率和精度上优于了现有的界面追踪方法,
更是为流相拓扑变化的分析及处理奠定了理论基础.

本文提出了linear MARS方法,
linear MARS方法将基于MARS理论的界面追踪法从二维推广到了三维.
二维cubic MARS为了最大化记录流相界面信息,
在调整步中保证了$C^1$不连续点不被删除;
在三维情况下,linear MARS的调整步中,
我们通过对局部的平整程度进行分析来限制调整操作,
从而进一步保护流相的界面信息.
另外,当流函数是同胚映射时,即使流相的形变很大,
linear MARS也能够保持其拓扑结构不改变.
最后,linear MARS允许用户在衡量计算精度和计算效率之后,
自行定义界面尺度和与控制体(即流相内部尺度)的关系,
为用户提供了更加灵活自由的接口.
\section{展望}
在本文中,我们用线性插值来增加示踪点,
用三角平面组成的多面体来近似三维空间上的殷集,
从而使得三维线性MARS方法的精度达到$2\alpha$阶.
此算法仅仅是MARS理论从二维推广的三维的第一步,仍存在许多不足,
需要进一步的改进与完善.对此算法的展望具体如下:
\begin{enumerate}
		 		\setlength{\itemsep}{0pt}
	\setlength{\parsep}{0pt}
	\setlength{\parskip}{0pt}
	\item 在 linear MARS中对于删除操作的限制,
	我们仅仅保持了其拓扑结果不发生变化.
	为了更进一步保持界面的信息,
	界面追踪中应该允许某些特定的点不被删除(如$C^1$不连续点).
	\item 完善各种流相界面的初始三角剖分的生成方法,
	将流相在流场中的实时变化可视化,
	以此来提升linear MARS方法和用户的交互能力.
	\item 考虑用样条曲面插值来添加示踪点,
	以一组样条曲面来近似三维空间上的殷集,
	使得界面追踪达到更高阶的精度.
	\item 另外,我们还可对三维MARS算法加入优化步,
	通过计算$\mathcal{M}^{n+1}\cap\mathcal{C}$求得局部解,
	以次来使得MARS方法可与高阶有限体积方法耦合,
	同时更精确有效地处理拓扑变化.
\end{enumerate}




